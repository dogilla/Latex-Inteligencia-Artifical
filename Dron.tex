\documentclass[10pt, a4paper]{article}
\usepackage[spanish, es-tabla]{}
\usepackage[utf8]{inputenc}
\usepackage{amsmath}
\usepackage{amssymb}
\usepackage{float} 
\usepackage[table,xcdraw]{xcolor}
\usepackage{booktabs}

\title{\begin{center}
 {\LARGE \scshape Universidad Nacional Aut\'onoma de M\'exico \\ Facultad de Ciencias }
  \rule{1\textwidth}{2.0pt}\\
\end{center}}
\author{
  Guzmán Mosco, Mario Alexis\\
  \and  
  Mart\'inez Mendoza, Miguel \'Angel\\
  \and
  Torres Bucio, Miriam\\
  \and
  LastName2, Carlos\\
}
\date{4 de Septiembre 2019}

\begin{document}

\maketitle

\section{Introduction}

La inteligencia Artificial abarca una gran variedad de subcampos, que van desde lo general a lo específico, como jugar ajedrez, probar teoremas matemáticos, escribir poesía, conducir un automóvil, por nombrar algunos ejemplos.

Teniendo 4 enfoques al momento de abordar un problema: Racional, Comportamiento, pensamiento y desempeño humano.

Pensando Racionalmente: Estudia las facultades mentales a través del uso de modelos computacionales.

Actuando Racionalmente: La inteligencia computacional es el estudio del diseño de agentes inteligentes.

Distintas disciplinas han ido contribuyendo al desarrollo y formalización de la Inteligencia Artificial. La filosofía, matemáticas, economía, Neurosciencia, Psicología, Ingeniería Computacional, Lingüística,teoría del control y cibernética.

La historia de implementación sobre IA es muy interesante y extensa, el primer trabajo que se reconoce fue realizado por Warren McCulloch y Walter Pitts (1943). Se basaron en 3 fuentes del conocimiento, filosofía básica, función de las neuronas en el cerebro y teoría computacional. Mostrando como ejemplo que cualquier función computable podría ser calculada por alguna red de neuronas conectadas y que todas las conectividades lógicas podría implementarse mediante estructuras de red simple.

El primer éxito comercial fue un sistema computacional llamado R1, programa que configura los sistemas informáticos. Dados el pedido de un cliente, determinar qué modificaciones, si es que hay alguna, se deben hacer al pedido por razones de funcionalidad del sistema y produce una serie de diagramas que muestran cómo se asociaron los diversos componentes del pedido. 

A mediados de los años 80 se tuvo el retorno de las redes neuronales, creando y aplicando algoritmos a problemas de aprendizaje en informática y psicología, la IA adoptó métodos científicos, en términos de metodología la IA ha quedado firmemente sujeta al método científico. Para ser aceptadas, las hipótesis deben ser sometidas a rigurosos experimentos empíricos, y los resultados deben ser analizados estadísticamente por su importancia. Ahora es posible replicar experimentos mediante el uso de repositorios compartidos de datos de prueba y código.

La aparición de agentes inteligentes, alentados en la resolución de subproblemas de la IA, los investigadores han comenzado a mirar nuevamente al problema del “agente completo”.
Uno de los entornos más importantes para la inteligencia es el internet

La disponibilidad de conjuntos de datos muy grandes, a lo largo de los 60 años de la informática le a dado énfasis en el algoritmo como tema principal de estudio. Pero algunos trabajos recientes en IA sugieren que para muchos problemas, tiene más sentido preocuparse por los datos y ser menos exigentes con respecto al algoritmo a aplicar.



\section{Definición del problema}
\subsubsection{Sistema basado en reglas}
Para el manejo del agente Dron se ha usado un sistema basado en reglas, el cual, permite manipular la información de forma util para los fines del agente, en palabras del profesor, "los sistemas basados en reglas son usados como una forma de almacenar y manipular el conocimiento einterpretarlo de una manera útil".\footnote{$ Sacado de: $ \\ $http://esie.icat.unam.mx/moodle/pluginfile.php/1684/mod_resource/content/1/Sistemas20Basados20en20Reglas.pdf$ }

En este caso el conocimiento que tenemos se trata del entorno y fue conseguido apriori  debido a que fue programado y contruido con el objetivo especifico de ser el espacio de movimiento del dron. Dado este contexto, se puede tener las siguientes hipotesis acerca del entorno y el movimiento del Dron:

\begin{enumerate}
\item El entorno es finito dado los límites establecido por el creador del entorno.
\item El espacio es lo suficientemente amplio como para que el dron.
\end{enumerate}

El sistema basado en reglas funciona en términos computacionales por medio de las expresiones "if", haciendo uso de su sémantica usual en cualquier lenguaje de programación.

Si la condicion se cumple $\Rightarrow$ realizar acción



\section{Solución del problema}
\begin{enumerate}
\item[$\clubsuit$]Que tipo de agente se propone utilizar?\\
El agente "Dron" implementado es este proyecto funciona en la práctica como un agente reactivo simple, es decir, cuenta con las siguientes caracteristicas: 
	\begin{enumerate}
	\item El ambiente, los sensores y los actuadores funcionan de manera conjunta sin presindir uno del otro
	\item Los sensores dan cuenta de como es el mundo y premiten tener un panorama del entorno 
	\item Los actuadores dictan las acciones a seguir. En nuestro caso, los actuadores se activan conforme a los sensores y éstos se activan guiados por las reglas dadas. 
\end{enumerate}

Estas caracteristicas fueron las que nos parecieron más pertinentes para el tipo de proyecto asignado por las siguientes razones: 
\begin{enumerate}
 \item Es un modelo relativamente más fácil de implementar computacionalmente haciendo uso de la programación orientada a objetos (en este caso C-Sharp. El trio actuadores-sensores-entornos se puede abstraer de una manera similar al pratrón de diseño de software modelo-vista-controlador siendo el entorno la vista; los actuadores el modelo y los sensores el controlador. Esto por supuesto es una analogía.
 
 \item Con respecto al punto anterior, nos parecio más sencilla la implementación siguiendo este modelo si lo comparamos respecto a la utilización de una maquina de estados. 
 
 \item Establecer de ésta manera el agente vuelve la solución al problema más mecanica y por tanto suceptible a ser resuelta de manera algoritmica.
 
 \item Las herramientas teoricas con las que contamos en el curso no nos permiten llegar a la solución de maneras más sofisticadas. 
 
\end{enumerate}
 

En la siguente tabla se ven esquematizados los sensores, actuadores y el ambiente del proyecto: 

\item[$\clubsuit$]Entorno de trabajo (Tabla $REAS$)
\begin{table}[h]
\centering
\begin{tabular}{|l|l|l|l|l|}
\hline
Agente & Rendimiento & Entorno & Actuadores & Sensores \\ \hline
Dron   &   \begin{tabular}[c]{@{}l@{}}Eficiente en tiempo \\ \\ Eficiente en ahorro de bateria \\ \\ Seguro \\ \\ Confiable\end{tabular}           &  \begin{tabular}[c]{@{}l@{}}Paredes \\ \\ Ciudad \\ \\ Bosque \\ \\ Montañas \\ \\ Clima \end{tabular}        &  \begin{tabular}[c]{@{}l@{}}h\'elices \\ \\ C\'amara \\ \\ Bater\'ia \\ \\ Motor \\ \\ Control\end{tabular} &  \begin{tabular}[c]{@{}l@{}} Radar \\ \\ Laser \\ \\ Giroscopio \\ \\ Sensor de bater\'ia \end{tabular}    \\ \hline
\end{tabular}
\caption{REAS}
\label{REAS}
\end{table}
\newpage

\item[$\clubsuit$]Propiedades del entorno (Tabla de entornos)
\begin{table}[h]
\centering
\begin{tabular}{|l|l|l|l|l|l|l|}
\hline
Entorno & Observable & \begin{tabular}[c]{@{}l@{}}Determinista/ \\ \\ Estocastico\end{tabular} & \begin{tabular}[c]{@{}l@{}}Episodico/ \\ \\ Secuencial\end{tabular} & \begin{tabular}[c]{@{}l@{}}Discreto/ \\ \\ Continuo\end{tabular} & \begin{tabular}[c]{@{}l@{}}Estatico/ \\ \\ Dinamico\end{tabular} & Agente     \\ \hline
\begin{tabular}[c]{@{}l@{}}Bloques/ \\ \\ Ciudad\end{tabular} & Totalmente & Determinista                                                         & Secuencial                                                          & Continuo                                                         & Estatico                                                         & Individual \\ \hline
\end{tabular}
\caption{ENTORNO}
\label{ENTORNO}
\end{table}
\item[$\clubsuit$]Percepciones y acciones






\item[$\clubsuit$]Percepciones y sensores (Describir como se relacionan)
\item[$\clubsuit$]Acciones y actuadores(Describir como se relacionan)
\item[$\clubsuit$]Funci\'on del agente (Que chingados hace XD)
Nuestro entorno al ser un conjunto de bloques estaticos nos pareció que una buena forma de lograr un recorrido a travéz del entorno sería aprovechar la geometría que implica su forma.
\end{enumerate}
\section{Conclusiones}
\begin{enumerate}
\item[$\clubsuit$]Ventajas y desventajas
\end{enumerate}
\section{Bibliografia}

Stuart Rusell and Peter Norvig, Artificial Intelligence: A Modern Approach', 3rd Edition, Prentice Hall, 2009 \\

Stephen Marsland, Machine Learning: An Algorithmic Perspective, Chapman y Hall, 2009 \\

Ernest Davis, Representations of Commonsense Knowledge, Morgan Kaufmann Pub, 1990 \\



\end{document}


