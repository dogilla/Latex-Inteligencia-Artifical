\documentclass{article}
\usepackage[utf8]{inputenc}
\usepackage{amsmath}

\title{Dron 2020-1}
\author{
  Guzman, Mosco\\
  Mario, Alexis\\
  \texttt{first1.last1@xxxxx.com}
  \and  
  LastName1, FirstName1\\
  \texttt{first1.last1@xxxxx.com}
  \and
  LastName2, FirstName2\\
  \texttt{first2.last2@xxxxx.com}
  \and
  LastName2, FirstName2\\
  \texttt{first2.last2@xxxxx.com}
}
\date{4 de Septiembre 2019}

\begin{document}

\maketitle

\section{Persepciones}
\subsection{Sistema basado en reglas}
Para el manejo del agente Dron se ha usado un sistema basado en reglas, el cual, permite manipular la información de forma util para los fines del agente, en palabras del profesor, "los sistemas basados en reglas son usados como una forma de almacenar y manipular el conocimiento einterpretarlo de una manera útil".
En este caso el conocimiento que tenemos se trata del entorno y fue conseguido apriori  debido a que fue programado y contruido con el objetivo especifico de ser el espacio de movimiento del dron. Dado este contexto, se puede tener las siguientes hipotesis acerca del entorno y el movimiento del Dron:

- El entorno es finito dado los límites establecido por el creador del entorno
- El espacio es lo suficientemente amplio como para que el dron 

Si topa obstaculo $\Rightarrow$ moverse



\end{document}


